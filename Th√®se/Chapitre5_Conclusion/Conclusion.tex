\chapter{Conclusions et perspectives}
%\addcontentsline{toc}{chapter}{Conclusion}
\label{chap:conclu}
%\newpage

\paragraph{} L'objectif principal de cette thèse était de développer des méthodes permettant de valoriser des données de crues historiques de formes variées pour l'analyse fréquentielle, avec une prise en compte complète et homogène des différentes sources d'incertitude. Même si l'utilisation de données historiques pour l'analyse fréquentielle des crues est une pratique courante, l'estimation et la propagation de l'ensemble des incertitudes est souvent négligée (A REFORMULER, TROP SÉVÈRE?). Pourtant, la prise en compte des incertitudes est particulièrement intéressante dans ce contexte pour plusieurs raisons. Tout d'abord, elle permet d'appréhender correctement l'étendue du risque de crue, ce qui peut être une aide à la prise de décision. Elle permet également de comparer raisonnablement les estimations provenant de différentes méthodes pour un même site. Enfin, elle permet également d'apporter des éléments de réponse à la question : \og \textit{jusqu'à quelle limite l'ajout de données historiques permet-t-elle d'améliorer l'estimation des quantiles extrêmes}\fg{}. 

\paragraph{} Dans les pages suivantes, nous résumons les diverses méthodes proposées dans ce manuscrit, ainsi que les résultats de leur application à la station du Rhône à Beaucaire, de 1500 à 2020. Enfin, nous discutons des perspectives qui découlent de ces résultats, au-delà du cadre de la thèse et du cas d'étude du Rhône à Beaucaire. 
	
	
	\section{Principaux résultats}
%	\addcontentsline{toc}{section}{Principaux résultats obtenus}
	
	\paragraph{} La première partie de la thèse consistait à effectuer un bilan des données de crue disponibles à Beaucaire. Cette étape ne doit pas être négligée car elle constitue les fondations de l'analyse fréquentielle. De plus, elle est tout particulièrement importante lorsque l'on s'intéresse aux incertitudes qui affectent ces données d'entrée. À Beaucaire, le précieux travail de recensement d'archives de \citet{pichard_les_1995} et \citet{pichard_hydro-climatology_2017}, ainsi que la récente étude de \citet{bard_actualisation_2018} ont permis de reconstituer une chronique continue de mesures limnimétriques débutant en 1816, ainsi que des jaugeages débutant en 1845. Ce chapitre a permis de retracer les nombreuses évolutions du chenal depuis le XIX\textsuperscript{ème} siècle, ce qui est particulièrement important afin de disposer de données objectives pour l'estimation des incertitudes hydrométriques. Une analyse de l'évolution des temps de propagation des crues (sélectionnées selon des critères bien précis) au cours des différentes phases d'aménagement en lit mineur a également été menée. Elle a permis de mettre en valeur une diminution de ces temps de propagation, débutée à la fin du XIX\textsuperscript{ème} siècle. Cette constatation pourrait impacter l'homogénéité des données hydrométriques de la période 1816-2020. L'étape suivante consistait à extraire et caractériser les données de crue de la base de données HISTHRÔNE (\url{histrhone.cerege.fr}) qui recense plus de 1500 événements hydroclimatiques depuis le XIII\textsuperscript{ème} siècle. Une estimation de l'ordre de grandeur du débit de ces crues, qui sont classées dans la base de données en différentes catégories basées sur les dommages, a été effectuée. Celle-ci se base sur une comparaison des crues de la base avec les hydrogrammes de la période récente (1816-2020). Ces estimations pourront permettre l'utilisation des crues historiques (antérieures aux mesures limnimétriques continues) pour l'analyse fréquentielle, car elles permettent une premier approche des seuils de perceptions. Cependant, des fluctuations de la vulnérabilité des populations aux inondations ont été mises en évidence et peuvent compliquer l'utilisation des seules mentions de crues, en l'absence d'estimations de débit pour chacun des événements. C'est pourquoi la piste de l'utilisation de modèles hydrauliques pour l'estimation du débit de chacune de ces crues a été explorée. Cependant, ce travail s'est heurté au manque de données, notamment pour reconstituer la bathymétrie et la rugosité du Rhône avant le XIX\textsuperscript{ème} siècle. Au regard des nombreuses incertitudes qui affectent cet exercice, et du temps limité de la thèse, nous avons décidé de ne pas poursuive ces travaux. Néanmoins, ils ont permis de définir les contours et les limites de la modélisation du débit des crues historiques, et de dégager des pistes de simplification du modèle hydraulique. 
	
	\paragraph{} La seconde partie de la thèse avait pour but d'explorer l'impact de la valorisation de données hydrométriques anciennes (et continues) sur l'analyse fréquentielle des crues. La première partie de ce travail consistait à élaborer les chroniques de débit à Beaucaire de 1816 à 2020 avec une prise en compte complète des incertitudes hydrométriques. Tout d'abord, les erreurs de mesure affectant les relevés de hauteur d'eau ont été estimées par la combinaison de diverses sources d'incertitude. Parmi ces sources, on peut notamment citer l'incertitude due à la fréquence des relevés, qui est particulièrement importante pour les mesures très anciennes. Cette incertitude a été estimée en dégradant artificiellement les données modernes effectuées à des pas de temps très fins par des capteurs automatiques. Les dates de détarage de la relation hauteur/débit ont ensuite été estimés à l'aide de la méthode de \citet{darienzo_detection_2021}, dont l'intérêt principal est de pouvoir considérer l'incertitude des jaugeages dans l'estimation de ces dates. Des courbes de tarage incertaines ont été déterminées pour chacune des périodes homogènes à l'aide de la méthode BaRatin SPD \citep{mansanarez_shift_2019}, basée sur l'interprétation physique des détarages. Cette méthode fait l'hypothèse que certains paramètres des courbes de tarage sont constants, alors que d'autres varient d'une période homogène à l'autre. Ainsi, l'information est transférée entre les périodes et permet de réduire l'incertitude des courbes de tarage. Les changements morphologiques identifiés au premier chapitre ont été utilisés pour la configuration de ce modèle. Ce sont ainsi 21 courbes de tarage incertaines qui ont été estimées. L'incertitude des relevés de hauteur d'eau a ensuite été propagée à travers ces courbes de tarage pour obtenir un hydrogramme incertain de 1816 à 2020. L'incertitude totale varie de 30\% au début du XIX\textsuperscript{ème} siècle, à 5\% pour les relevés les plus récents. Cette incertitude hydrométrique a ensuite été propagée jusqu'aux quantiles extrêmes estimés selon une distribution GEV. Ainsi, on peut déterminer la contribution de l'incertitude hydrométrique et de l'incertitude d'échantillonnage parmi l'incertitude totale des quantiles extrêmes. Pour la chronique complète (205 ans), l'incertitude hydrométrique est dominante pour des périodes de retour inférieures à 100 ans, alors que l'incertitude d'échantillonnage est dominante au-delà de la centennale. Les estimations GEV ont ensuite été réalisées pour des durées de chroniques variables, en sous-échantillonnant dans la chronique complète. Il apparait que l'incertitude totale diminue significativement pour des durées de chroniques allant de 20 à 100 ans. Au-delà de 100 ans, l'incertitude est très stable. Cependant, les estimations maxpost augmentent d'environ 15\% pour des durées de chroniques supérieures à 160 ans, et ce à cause de l'inclusion des deux plus fortes crues de l'échantillon en 1840 et 1856. Ces résultats ont permis d'illustrer l'intérêt des relevés hydrométriques anciens pour la réduction des incertitudes des quantiles de crue, mais également d'identifier la part de chacune de sources d'incertitude dans cette réduction. Ils ont également permis de démontrer l'importance d'une estimation complète et homogène des incertitudes pour l'analyse fréquentielle des crues. 
	
	\paragraph{} Les résultats obtenus pour la chronique continue de 1816 à 2020 ont ouvert la voie à l'inclusion de données historiques, antérieures aux enregistrements continus et extraites de la base HISTRHÔNE au premier chapitre. En l'absence de reconstructions du débit de chacun des événements, le nombre d'occurrences de crues supérieures à un seuil de perception sera exploité. Ce seuil de perception découle des catégories de crues de la base HISTRHÔNE, il n'a pas un sens physique direct mais est plutôt lié à la perception des dommages. Si l'utilisation d'un échantillon mixte (composé de données hydrométriques continues et de données de crues historiques ponctuelles) est une pratique d'analyse fréquentielle relativement commune, la propagation complète des incertitudes hydrométriques et des incertitudes d'échantillonnage est souvent négligée. De plus, le seuil de perception et la durée historique durant laquelle ce dernier est actif sont dans la grande majorité des cas supposés parfaitement connus, ce qui est souvent une forte hypothèse. Le modèle proposé dans ce chapitre les considère comme des paramètres à part entière du modèle probabiliste. Ainsi, la méconnaissance du seuil de perception et de la durée de la période historique se reflète dans l'incertitude des résultats. Dans un premier temps, ce modèle a été appliqué à la chronique de débits continue du Rhône à Beaucaire de 1816 à 2020, qui a été artificiellement dégradée en un échantillon mixte. Cet exercice permet de tester le modèle sur un cas d'étude pour lequel seuil de perception et durée de la période historique sont parfaitement connus. Les estimations obtenues sont satisfaisantes, même si elle ne permettent évidemment pas aussi précises que les estimations du chapitre précédent qui utilisent l'entièreté de la chronique continue. Ces tests ont permis d'identifier que la seule méconnaissance du seuil de perception entrainait une incertitude bien plus grande que la seule méconnaissance de la durée de la période historique. En revanche, quand ces deux paramètres sont considérés incertains en même temps, l'incertitude autour des quantiles est réduite par rapport au cas où seul le seuil est incertain. Ce résultat s'explique par une corrélation entre ces deux paramètres. Ces premiers résultats démontrent également que considérer que le seuil de perception est parfaitement connu lorsque ce n'est pas le cas peut mener à une sous-estimation importante de l'incertitude des résultats. Le modèle a ensuite été appliqué aux données complètes du Rhône à Beaucaire, avec des occurrences de crues historiques pour la période 1500-1815, et la chronique de débits maximum annuels de 1816 à 2020. Les résultats présentent une incertitude réduite par rapports aux résultats de la seule chronique continue de 1816 à 2020, et ce même dans le cas où seuil de perception et durée de la période
historiques sont incertains (à condition d'utiliser des a priori suffisamment informatifs). Cependant, cet exercice a permis de mettre en lumière une probable sous-estimation du nombre de dépassements du seuil de perception au cours de la période historique dans les données HISTRHÔNE. Cette potentielle non-exhaustivité apparait alors même que l'homogénéité des données a été validée. Elle pourrait provenir du fait que les catégories de la base HISTRHÔNE sont définies sur la perception des dommages par les populations ripariennes, et non sur un seuil de perception physique directement lié au dépassement d'un débit. Même si le cas d'étude de Beaucaire de 1500 à 2020 présente des limites qui semblent difficiles à surmonter, le modèle proposé dans ce chapitre ouvre la porte à une prise en compte complète et homogène des incertitudes dans le cas de l'analyse fréquentielle des crues historiques. 
	
	\paragraph{ Paragraphe supplémentaire pour conclure plus généralement ?}
	Au-delà de l'analyse fréquentielle, les données historiques de crues représentent un patrimoine précieux, que ce soit pour l'analyse de la variabilité climatique long terme, ou tout simplement pour garder dans la mémoire collective l'existence de ces événements exceptionnels. Même si le Rhône peut aujourd'hui paraitre très aménagé, endigué, voire même dompté, des événements dévastateurs restent possibles. La crue de décembre 2003 a fait ressurgir des peurs au sein des populations Rhodaniennes, qui au fil des générations avaient oublié la gravité des crues de 1840 et 1856. C'est en ce sens que les initiatives de data-rescue similaires au projet HISTRHÔNE sont d'un grand intérêt, à la fois pour la science, mais également pour la sensibilisation collective aux risques naturels. En effet, le plus gros risque pour les populations ne provient peut-être pas des crues elles-même, mais plutôt des mécanismes d'oubli qui interviennent au fil des générations.
	
	\paragraph{} Court hommage à G. PICHARD quelque part ? Après les perspectives ou dans les remerciements ?
	
	\section{Perspectives des travaux de thèse}
%	\addcontentsline{toc}{section}{Perspectives des travaux de thèse}
	
	\paragraph{Modélisation du débit des crues historiques}
	Un modèle pour chaque crue histo ? gros boulot et pas assez de données\\
	Détermination des incertitudes de la modélisation par des scénarios haut/bas sur chacun des paramètres/données d'entrée. Ou bien encore mieux, propagation des incertitudes par simulations Monte Carlo du modèle MAGE. Mais gros temps de calcul vu les nombreuses sources d'incertitude\\
	Simulations en régime non-permanent pour mieux appréhender l'impact des ruptures de digues sur le débit de pointe. Mais là aussi, manque de données\\
	Courbe de tarage Arles/Beaucaire basée sur les hauteurs à l'échelle reconstituée de Véran. Piste intéressante, mais attention aux ruptures de digues entre Beaucaire et Arles.\\
	Dans le cas ou le débit des crues historiques serait connu, QUID du cas où l'incertitude du seuil de perception et l'incertitude du débit de certaines crues se chevauche? Inférieure ou supérieure au seuil ?\\
	De plus, dans le cas de Beaucaire, même si le débit des crues est modélisé, cela ne règle en aucun cas les potentiels problèmes de non-exhaustivité\\
		
	\paragraph{Etude de la variabilité des débits 1816-2020}
	Analyse de la forme des hydrogrammes de crue, CF stage Raphaelle\\
	Analyse des évolutions de la saisonalité des crues\\
	Les quantiles estimés sur 1816-2020 sont utilisés comme "référence", mais cette période est celle durant laquelle le BV a subi les altérations les plus rapides et les plus importantes (chenalisation, digues, artificialisation des sols, modification). Cela peut expliquer le déséquilibre de la fréquence des crues > S4 pour la période récente par rapport à la période ancienne, même en l'absence de ruptures dans les données.\\
	Est-ce que le problème vient vraiment d'une non-exhaustivité des données historiques ? Et si y avait une rupture dans la distribution des crues à partir de 1816?
	
	\paragraph{Segmentation des jaugeages} Certains détarages ont pu être ratés à cause du faible nombre de jaugeages anciens : utilisation d'autres méthodes, par exemple les récessions ? Mais c'est certainement moins fiable quand le pas de temps des mesures limni est très grand. 
	
	\paragraph{Elicitation des a priori des CT} Pas si simple pour la période ancienne, mais comment faire mieux ? Modèle hydraulique ?\\
	
	\paragraph{Analyse fréquentielle non-stationnaire}
	Le cas de Beaucaire est particulier car la chronique est très longue. Même si on avait 500 ans de mesures de hauteur d'eau en continu, on aurait certainement des problèmes dus à la variabilité climatique. Les problèmes rencontrés à Beaucaire ne sont pas dus qu'à la nature des données (témoignages)\\
	Prise en compte de co-variables climatiques d	ans le modèle, lesquelles ? \\
	Broninmann, Hidden climate indices Ben, autres? \\
	Analyse conditionnée aux types de temps, CF SHADEX\\
	Ne pas oublier que la chronique peut servir à étudier la variabilité des crues sur le long terme, au delà de la seule analyse fréquentielle.
	
	\paragraph{Analyse historique}
	Difficile de généraliser les résultats obtenus en dehors de Beaucaire : paramètre de forme positif et grand BV. Par exemple, on aurait certainement eu des résultats différents pour un petit BV avec une queue de distribution plus lourde. \\
	Il serait intéressant de tester la méthode (avec S et n incertains) sur des données historiques de nature différente (repères de crues, dendrochronologie,...)\\
	QUID d'un modèle où non seulement le seuil et la durée sont des paramètres, mais également le nombre d'occurrences de crues supérieures au seuil (k). C'est envisageable et un peu plus compliqué, mais sûrement pas très utile vu l'incertitude que ça doit engendrer\\
	QUID d'un modèle avec plusieurs seuils incertains, et dont la durée est également incertaine ? C'est peut être un modèle un peu trop compliqué\\
	Peut être qu'au lieu de passer du temps à complexifier les modèles, il vaut mieux passer du temps sur la recherche et la caractérisation des données histo.
	
	\paragraph{Analyse régionale} même si BCR est un très grand BV, il est peut être possible de faire de l'analyse régionale, notamment avec des stations du Rhône plus à l'amont.\\
	Piste explorée avec Ben
	
	\paragraph{Plus généralement} Il serait sans doute intéressant d'avoir une telle méthode de propagation des incertitudes de A à Z qui soit rendue opérationnelle et utilisable sur plein de stations\\
	Mais vu le nombre d'étapes (segmentation, élicitation des a priori CT, modèle d'erreur sur les hauteurs, etc), pas dit que ça soit facile de faire une méthode vraiment généralisable car ça demande une expertise quasiment à chaque étape.	
	
%	méthode opérationnelle, transparente (outils open-source), documentée et justifiée, transférable à des utilisateurs extérieurs et bien sûr généralisable à d'autres sites et contextes.
	
%	\paragraph{} pour finir, citation eysette +	le plus gros risque ce n'est pas la crue mais l'oubli


