\thispagestyle{empty}

\noindent \textbf{Abstract:}

\noindent \small{Statistical estimation of flood risk generally involves estimating the parameters of a distribution based on data from hydrometric stations. This exercise is affected by important uncertainties that come from the accuracy of the available data, but also from the limited size of the stage series. In order to reduce the uncertainty, it is possible to expand the dataset by using infrequent flood data older than the installation of stream gauges. The use of these data requires the assumption that all floods above a certain magnitude have left a record. This magnitude, called the perception threshold, is sometimes complex to determine, as well as the historical periode length. However, it is generally considered to be perfectly known. Moreover, the estimation and propagation of each of the sources of uncertainty is rarely performed. The main objective of this thesis is to develop a frequency analysis method that allows the valuation of historical continuous or infrequent data, with a complete and homogeneous consideration of the different sources of uncertainty. One of the original features of this method is the inclusion of the perception threshold and the duration of the historical period as parameters of the probabilistic model, using a Bayesian approach. This method is applied to the exceptional case study of the hydrometric station of the Rhône River at Beaucaire, France, for which more than 200 years of continuous stage records, but also an heritage of hydroclimatic data going back to the XIII\textsuperscript{th} century are available. A continuous streamflow series with uncertainty was estimated from 1816 to 2020. This uncertainty was propagated to the design flood estimates, allowing to quantify the contribution of both hydrometric and sampling uncertainties to the total uncertainty. Tests carried out for variable series lengths allowed to identify that the total uncertainty is decreasing significantly when the length of the series increases from 20 to 100 years. Beyond 100 years, the total uncertainty remains relatively stable. The reconstructed hydrograph was then artificially degraded to reproduce a mixed sample of continuous and infrequent data, in order to test the model on a perfectly known case. This test identified that the lack of knowledge about the perception threshold resulted in a much greater uncertainty than the lack of knowledge on the duration of the historical period. The model was then applied to historical floods since the XVI\textsuperscript{th} century. The results show a reduced uncertainty compared to the results of the only continuous record from 1816 to 2020, even in the case where the perception threshold and the duration of the historical period are considered uncertain. This model is satisfactory but its application has detected probable gaps in the historical data. Beyond design flood estimation, the analysis of climate variability over more than five centuries in Beaucaire is an encouraging perspective.}

\vfill

\noindent \textbf{Résumé :}

\noindent \small{L'estimation statistique du risque de crue consiste généralement à estimer les paramètres d'une distribution en se basant sur les chroniques de débit. Cet exercice est affecté par des incertitudes importantes qui proviennent de la précision des données utilisées, mais également de la taille limitée des chroniques de débit. Afin d'améliorer les estimations des quantiles de crue, il est possible d'élargir temporellement le jeu de données en utilisant des données de crue ponctuelles antérieures à l'installation des réseaux de mesure. L'utilisation de ces données nécessite  de faire l'hypothèse que toutes les crues ayant dépassé une certaine magnitude ont laissé une trace. Cette magnitude, appelée seuil de perception, est parfois complexe à déterminer, tout comme la durée de la période historique. Pourtant, elle est généralement considérée comme étant parfaitement connue. De plus, l'estimation et la propagation de chacune des sources d'incertitude est rarement effectuée. L'objectif principal de la thèse est de développer une méthode d'analyse fréquentielle permettant de valoriser des données anciennes, qu'elles soient continues ou ponctuelles, avec une prise en compte complète et homogène des différentes sources d'incertitude. Une des originalités de cette méthode provient de l'inclusion du seuil de perception et de la durée de la période historique comme étant des paramètres du modèle probabiliste à l'aide d'une approche bayésienne. Cette méthode est appliquée au cas d'étude exceptionnel de la station hydrométrique du Rhône à Beaucaire, pour lequel plus de 200 ans de relevés continus de hauteur d'eau, mais également un patrimoine de données hydroclimatiques remontant au XIII\textsuperscript{ème} siècle sont disponibles. Une chronique continue de débits avec incertitudes a été estimée de 1816 à 2020. Cette incertitude a été propagée aux estimations des quantiles de crue, permettant ainsi de quantifier la part de l'incertitude hydrométrique et de l'incertitude d'échantillonnage dans l'incertitude totale. Des tests réalisés pour des durées de chroniques variables ont permis d'identifier que l'incertitude totale diminuait significativement lorsque cette durée augmentait entre 20 et 100 ans. Au-delà, l'incertitude est relativement stable. La chronique de débits a ensuite été artificiellement dégradée pour reproduire un échantillon mixte composé de données continues et ponctuelles, dans le but de tester le modèle sur un cas parfaitement connu. Cela a permis d'identifier que la méconnaissance du seuil de perception entrainait une incertitude bien plus grande que la méconnaissance de la durée de la période historique. Le modèle a ensuite été appliqué aux crues historiques depuis le XVI\textsuperscript{ème} siècle. Les résultats présentent une incertitude réduite par rapport aux résultats de la seule chronique continue de 1816 à 2020, et ce même dans le cas où seuil de perception et durée de la période historique sont considérés incertains. Le modèle est satisfaisant et son application a permis de détecter de probables lacunes dans les données historiques à Beaucaire. Au-delà de l'analyse fréquentielle, l'analyse la variabilité climatique sur plus de cinq siècles constitue une perspective encourageante.
}