% Encodage des caractères et langue du document
\usepackage[T1]{fontenc}
\usepackage[utf8]{inputenc}
\usepackage{lmodern}
\usepackage[french]{babel}

\setcounter{tocdepth}{2} % Pour que les subsubsections n'apparaissent pas dans la TOC
\setcounter{secnumdepth}{2} % Pour que les subsubsections ne soient pas numérotées

%\usepackage[backend=biber,style=authoryear,bibstyle=authoryear,natbib=true,
%giveninits=true,uniquename=false,uniquelist=false,maxcitenames=2,date=year,
%maxbibnames=99,url=false]{biblatex}

% Gestions des marges
\usepackage{geometry} 
\geometry{a4paper}
%\geometry{margin=2.5cm}
%\geometry{headheight=6mm}

% Si on a besoin d'une configuration plus précise des marges
\geometry{%
  a4paper,                % format de papier
% Définition des marges :
  left= 3cm,            % marge intérieure à la page
  right = 2cm,          % marge extérieure
  top = 3cm,
  bottom = 3cm,
% En-tête et pied de page :
  headheight=6mm,         % espace réservé à l'en-tête dans la marge top
  %headsep=3mm,            % espace entre le corps et l'en-tête
  %footskip=9mm            % espace entre le corps et le pied de page
}

% Pour une meilleure gestion des maths
\usepackage{amsmath,amssymb,amsfonts,amsthm}
%\usepackage{mathtools} % version modifiée de amsmath, ajoute des symboles, etc.
\usepackage{mathrsfs}% pour rajouter un format de lettres façon "caligraphie" en math mode.

% Pour gérer les unités
%\usepackage{siunitx}
%\sisetup{inter-unit-product=\ensuremath{{}\cdot{}}} % pour mettre des points médians entre les unités quand il y en a plusieurs
%\sisetup{separate-uncertainty=true,multi-part-units=single} % pour faire des incertitudes en écrivant \SI{valeur(incertitude)}{unité}
%\DeclareSIUnit\vitesse{\meter\per\second}


% Pour une meilleure gestion des graphiques
\usepackage{graphicx}
\usepackage{caption}
\usepackage{subcaption} % permet de faire des subfigures (remplace le package subfig)

% Pour les tableaux, matrices, listes
\usepackage{booktabs}
\usepackage{array}
\usepackage{paralist} % si besoin regarder "enumitem" qui est un peu différent
%\usepackage{multirow} % pour faire des tableaux plus compliqués

% Pour personnaliser les entêtes et pieds de pages
\usepackage{fancyhdr}
\usepackage{emptypage} % garantit que les pages blanches avant les débuts de chapitres soient vraiment blanches (pas d'entête ni de pied de page)
 
 % On définit le style pour les pages "normales"
\pagestyle{fancy}
\renewcommand{\chaptermark}[1]{\markboth{\chaptername \ \thechapter.\ #1}{}} % sert à personnaliser l'affichage de \leftmark (ici : le mot "Chapitre", le numéro, un point, et le titre du chapitre, sans écrire en majuscules)
\renewcommand{\sectionmark}[1]{\markright{\thesection.\ #1}} % sert à personnaliser l'affichage de \rightmark (ici : le numéro et le titre de la section en cours, sans écrire en majuscules)
\fancyhf{} % assure que les entête et pieds de page sont vides au départ
\fancyhead[LE]{\leftmark}
\fancyhead[RO]{\rightmark}
\fancyfoot[LE,RO]{\thepage}

% On définit le style pour les pages "spéciales" (début de chapitre, table des matières, etc.)
\fancypagestyle{plain}{
\fancyhf{}
\fancyfoot[RO,RE]{\thepage}
\renewcommand{\headrulewidth}{0pt}
\renewcommand{\footrulewidth}{0pt}}

% Explications :
% L = left, R = right, C = center, E = even pages, O = odd pages
%\thepage : adds number of the current page.
%\thechapter : adds number of the current chapter.
%\thesection : adds number of the current section.
%\chaptername : adds the word "Chapter" in English or its equivalent in the current language.
%\leftmark : adds name and number of the current top-level structure (for example, Chapter for reports and books classes; Section for articles ) in uppercase letters.
%\rightmark : adds name and number of the current next to top-level structure (Section for reports and books; Subsection for articles) in uppercase letters.

% Pour personnaliser les premières pages des chapitres
\usepackage[Lenny]{fncychap}

% Pour la table des matières
%\usepackage[nottoc]{tocbibind} % pour que la bibliographie apparaisse dans la table des matières (avec l'option pour que la table des matières elle-même n'apparaisse pas dans la table des matières).
\usepackage{tocloft}% pour pouvoir modifier les tailles d'espacement dans la table des matières
%\setlength\cftaftertoctitleskip{10pt} % pour fixer la taille après le titre "table des matières" et la table des matières en elle-même.

% Divers
\usepackage{textcomp} % rajoute des symboles (notes de musiques, etc.)
\usepackage{xcolor} % pour ajouter de la couleur (si besoin)
\usepackage{epigraph} % pour rajouter des citations en début de chapitre
% Exemple :
%\epigraph{Citation}}{Auteur}
%\usepackage{cite} % pour faire citations de façon pratique (notamment mettre [1-3] au lieu de [1,2,3]).
%\usepackage{quotchap} % pour rajouter des citations en début de chapitre (change la présentation des chapitres)
% Exemple :
%\begin{savequote}
%Citation
%\qauthor{Auteur}
%\end{savequote}

% Pour gérer les liens dans le document et avoir les signets dans le PDF
%\usepackage{hyperref}
\usepackage{bookmark}

% Configuration de hyperref
\definecolor{linkcolor}{rgb}{0,0,0.6} % couleur des liens (bleu foncé)
%\hypersetup{
%	colorlinks=true, % colore les liens au lieu de les encadrer
%	pdfstartview=FitV, % ouvre le PDF de façon à ce qu'il prenne la taille verticale de l'écran
%	urlcolor=linkcolor, % choix de la couleur des liens URL
%	linkcolor= linkcolor, % choix de la couleur des liens internes (table des matières, etc.)
%	citecolor=linkcolor % choix de la couleur des liens de citations
%}


%PACKAGES MATHIEU
\usepackage[backend=biber,style=authoryear,bibstyle=authoryear,natbib=true,
giveninits=true,uniquename=false,uniquelist=false,% firstinits=false,
maxcitenames=2,date=year, maxbibnames=99,url=false]{biblatex}
%\usepackage[pdftex]{graphicx}
\usepackage{tabularx}
\usepackage{dsfont}
\usepackage{multirow}
%\usepackage{amsmath,amsfonts,amssymb}
%\usepackage{subcaption}
\usepackage{authblk}
%hyperlinks options
\usepackage{hyperref}
\hypersetup{colorlinks=true,linkcolor=blue,filecolor=magenta,urlcolor=cyan,citecolor=cyan}
%\geometry{left=20mm, top=20mm, right=20mm}
\usepackage{placeins}
\usepackage{pdfpages}