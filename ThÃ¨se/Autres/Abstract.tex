\thispagestyle{empty}

\noindent \textbf{Abstract:}

\noindent TEXTE OF THE ABSTRACT 

\vfill

\noindent \textbf{Résumé :}

\noindent \small{L'estimation statistique du risque de crue, aussi appelée analyse fréquentielle, est nécessaire pour la protection des populations et des enjeux à proximité des cours d'eau. L'analyse fréquentielle des crues consiste généralement à estimer les paramètres d'une distribution en se basant sur les données des stations hydrométriques. Cet exercice est affecté par des incertitudes importantes qui proviennent des données utilisées, mais également de la taille limitée des chroniques de débit. Afin d'améliorer les estimations des quantiles de crue, il est possible d'élargir temporellement le jeu de données en utilisant des données historiques antérieures à l'installation des réseaux de mesure. Cependant, l'estimation et la propagation de chacune des sources d'incertitude qui affectent cet exercice est rarement effectuée. L'objectif principal de ce travail de thèse est de développer une méthode opérationnelle pour l'analyse fréquentielle des crues permettant de valoriser des données de crue anciennes avec une prise en compte complète et homogène des différentes sources d'incertitude. Cette méthode a été appliquée au cas d'étude exceptionnel de la station hydrométrique du Rhône à Beaucaire, pour lequel plus de 200 ans de relevés continus de hauteur d'eau sont disponibles, mais également un patrimoine de données hydroclimatiques remontant au XII\textsuperscript{ème} siècle. 

}