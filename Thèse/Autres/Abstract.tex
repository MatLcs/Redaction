\thispagestyle{empty}

\noindent \textbf{Résumé :}

\noindent \small{L'estimation statistique du risque de crue consiste généralement à estimer les paramètres d'une distribution en se basant sur les séries temporelles de débit. Cet exercice est affecté par des incertitudes importantes qui proviennent de la précision des données utilisées, mais également de la longueur limitée des séries de débit. L'objectif principal de la thèse est de développer une méthode d'analyse fréquentielle permettant de valoriser des données anciennes, qu'elles soient continues ou ponctuelles, avec une prise en compte complète et homogène des différentes sources d'incertitude. Elle est appliquée au cas d'étude exceptionnel de la station hydrométrique du Rhône à Beaucaire (95 590 km²), où plus de 200 ans de relevés continus de hauteur d'eau, mais également un patrimoine de données hydroclimatiques remontant au XIII\textsuperscript{ème} siècle sont disponibles. Une série continue de débits avec incertitudes de 1816 à 2020 a été estimée. Cette incertitude hydrométrique a été propagée aux estimations des quantiles de crue, permettant ainsi de quantifier la part de l'incertitude hydrométrique et celle de l'incertitude d'échantillonnage dans l'incertitude totale. Des tests réalisés pour des durées de séries variables ont permis d'identifier que l'incertitude totale diminue significativement lorsque cette durée augmente entre 20 et 100 ans. Au-delà, l'incertitude est relativement constante car la diminution de l'incertitude d'échantillonnage est compensée par l'augmentation de l'incertitude hydrométrique. Afin de réduire l'incertitude d'échantillonnage, le jeu de données a été élargi en utilisant des données de crue ponctuelles, antérieures à l'installation de la station hydrométrique. Une des originalités de cette méthode provient de l'inclusion du seuil de perception et de la durée de la période historique comme étant des paramètres du modèle probabiliste, à l'aide d'une approche bayésienne. Le modèle a été testé sur la série de débits continue, artificiellement dégradée, et pour laquelle seuil de perception et durée de la période historique sont donc parfaitement connus. Cela a permis d'identifier que la méconnaissance du seuil de perception entrainait une incertitude bien plus grande que la méconnaissance de la durée de la période historique. Le modèle a ensuite été appliqué aux crues historiques depuis le XVI\textsuperscript{ème} siècle. Les résultats présentent une incertitude réduite par rapport aux résultats de la seule série continue de 1816 à 2020, et ce même dans le cas où seuil de perception et durée de la période historique sont considérés incertains. Néanmoins, une probable non-exhaustivité des données historiques a été détectée, ce qui complexifie l'utilisation de ces résultats. Au-delà du cas particulier de Beaucaire, il serait intéressant d'appliquer cette méthode sur des bassins versant pour lesquels le contexte climatique et la nature des données historiques sont différents. De plus, la longue série reconstituée à Beaucaire pourrait également être utilisée pour étudier la variabilité hydroclimatique du Rhône.}

\vfill

\noindent \textbf{Abstract:}

\noindent \small{ The statistical estimation of flood risk generally consists of estimating the parameters of a distribution based on streamflow time series. This exercise is affected by substantial uncertainties that come from the accuracy of the available data, but also from the limited length of the records. The main objective of this thesis is to develop a flood frequency analysis method that makes the most of continuous or sporadic historical data with a complete and homogeneous consideration of the various sources of uncertainty. The method is applied to the exceptionally rich case study of the Rhône River at Beaucaire, France (95 590 km²), with continuous stage records over more than 200 years and a comprehensive dataset on hydroclimatic events from the XIII\textsuperscript{th} century. A continuous streamflow series with uncertainties from 1816 to 2020 was established. The hydrometric uncertainty was propagated to the design flood estimates, and the contribution of both hydrometric and sampling uncertainties to the total uncertainty was quantified. Tests showed that the total uncertainty decreases significantly when the length of the series increases from 20 to 100 years. Beyond 100 years, the total uncertainty remains constant because the sampling uncertainty decrease is offset by the hydrometric uncertainty increase. To reduce the sampling uncertainty, the dataset was expanded by using sporadic historical flood data, prior to the hydrometric records. An original feature of this method is the inclusion of the perception threshold and the length of the historical period as parameters of the probabilistic model, using a Bayesian approach. The model was tested on the subsampled 1816-2020 streamflow series, for which the perception threshold and historical period length are known. The imperfect knowledge of the perception threshold resulted in a much greater uncertainty than the imperfect knowledge of the length of the historical period. The model was then applied to the full dataset with historical floods since the XVI\textsuperscript{th} century. The design flood uncertainty was smaller than using 
the continuous 1816-2020 streamflow time serie only, even when the perception threshold and the length of
the historical period are considered uncertain. Nevertheless, this application suggests that the historical data are probably incomplete, which complicates the use of these results. Beyond the specific case of Beaucaire,
it would be interesting to apply this method to watersheds for which the climatic context and the nature of the
historical data are different. Moreover, the long series reconstructed in Beaucaire could also be used to study the hydroclimatic variability of the Rhône River.}

