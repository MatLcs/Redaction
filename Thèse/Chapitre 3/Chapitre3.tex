\documentclass[11pt]{article}
% packages
\usepackage[utf8]{inputenc}
\usepackage{geometry}
\usepackage[pdftex]{graphicx}
\usepackage{tabularx}
\usepackage{dsfont}
\usepackage{multirow}
\usepackage{amsmath,amsfonts,amssymb}
\usepackage{subcaption}
\usepackage{authblk}
%hyperlinks options
\usepackage{hyperref}
\hypersetup{colorlinks=true,linkcolor=blue,filecolor=magenta,urlcolor=cyan,citecolor=cyan}
%bib options
\usepackage[backend=biber,style=authoryear,bibstyle=authoryear,natbib=true,
giveninits=true,uniquename=false,uniquelist=false,% firstinits=false,
maxcitenames=2,date=year, maxbibnames=99,url=false]{biblatex}
\geometry{left=20mm, top=20mm, right=20mm}
%float barrier
\usepackage{placeins}
 \addbibresource{Thèse.bib}
\title{Chapitre 3 : Analyse probabiliste des crues du XVI\textsuperscript{ème} siècle à aujourd'hui}
\author{Mathieu}


\begin{document}
\maketitle

\tableofcontents

\section{Introduction du chapitre}
	Quelques rappels d'intro aux méthodes d'analyse des crues histo:
	
	Solutions quand le débit est connu\\
	Solution quand le débit n'est pas connu	\\
	plot positions (\citet{hirsch_techniques_1982}...)	\\
	seuil de perception\\
	durée de la période historique\\
	Données dispo à Beaucaire
	
	
\section{Homogénéité des données disponibles à Beaucaire}

	\subsection{Données continues (1816-2020)}
		Tests de Mann-Kendall et Pettit
		
	\subsection{Données historiques censurées (1500-1816)}
		Description des 2 échantillons ("C3 \& C4" et "C4") et test de \citet{lang_towards_1999}
		
\section{Méthodes d'analyse probabiliste d'un échantillon mixte de crues}
	
	Description de diverses méthodes de la littérature
	
	\subsection{Concepts de base et hypothèses}
	Explication du concept de seuil de perception et de ses limites.\\
	Complexité de déterminer le seuil\\
	Complexité de déterminer la durée de la période historique \citep{prosdocimi_german_2018}). Attention aux confusions entre date de début de la période et durée de la période. Clarifier dès le début. Quel est le bon terme en français? En anglais d'après \citet{prosdocimi_german_2018}: period of time covered by the historical events or historical period coverage\\
	
	\paragraph{}
	Description d'un modèle "simple" GEV + censure binomiale (censure binomiale est le bon terme?):\\
	La probabilité d'observer une occurrence inférieure au seuil de perception est égale à 1 -  la probabilité de dépassement\\
	Les occurrences au-dessus d'un seuil $x$ peuvent être représentées par une binomiale dont la probabilité de succès (i.e. un dépassement) $p$ est égale à 1 - la fonction de répartition de la GEV ($e^{-t(x)}$) où $x$ est supposé égal au seuil de perception, soit $\mathcal{B}(n,p)$ où:
	\begin{equation}
		p = 1 - e^{ -(1+\xi ((x-\mu)/\sigma))^{-1/\xi} }
	\end{equation}
	et $n$ est égal aux nombre d'années de la période historique.\\
	
	\subsection{Propagation des incertitudes hydrométriques de la période continue}
	Description du modèle (A): \\
	Idem au modèle décrit juste avant, avec ajout de la propagation des incertitudes hydrométriques de la période continue par MCMC\\
	Procéder de la même manière que dans l'article JoH
	
	\subsection{Prise en compte des incertitudes autour du seuil de perception}
	Description du modèle (B)\\
	idem au modèle (A) mais $x$ le seuil de perception n'est plus fixe, une distribution a priori lui est affectée. 
		
	\subsection{Prise en compte des incertitudes autour du nombre d'années de validité du seuil de perception}
	Description du modèle (C)\\
	Idem modèle (B), mais le nombre d'années de validité du seuil $n$ n'est plus fixe, une distribution a priori lui est affectée.\\
	
	
	\subsection{Tableau qui présente les 3 modèles utilisés (ABC) et le cas "Baseline"}
	Modèle (Baseline): GEV sur les max annuels (uniquement sur la période continue) avec propagation des incertitudes hydrométriques.\\
		Modèle (A): GEV et censure binomiale, le seuil de perception et le nombre d'année sont fixes, propagation des incertitude hydrométriques de la période continue.\\
		Modèle (B): GEV et censure binomiale, le seuil de perception est un paramètre du modèle, le nombre d'années est fixe, propagation des incertitude hydrométriques de la période continue.\\
		Modèle (C): GEV et censure binomiale, le seuil de perception et le nombre d'années sont des paramètres du modèle, propagation des incertitude hydrométriques de la période continue.\\
		
	
\section{Application aux crues du Rhône à Beaucaire}	
	\subsection{Estimation des quantiles de crues 1500-2020}
	Résultats du modèle (A) pour l'échantillon ("C3 \& C4")
	
	\subsection{Quel est l'apport des crues historiques pour l'analyse fréquentielle à Beaucaire ?}
	Comparaison des résultats du modèle (Baseline) avec les résultats du modèle (A) pour l'échantillon ("C3 et C4").\\
	Barplot Q100 et Q1000
	
	\subsection{Quel est l'impact du choix de l'échantillon de crues historiques ("C3etC4" vs "C4")}
	Comparaison des résultats du modèle (A) pour les échantillon ("C3 \& C4") et ("C4")\\
	Barplot Q100 et Q1000	
	
	\subsection{Quel est l'impact de la méconnaissance du seuil de perception ?}
	Comparaison des résultats du modèle (A) et du modèle (B) pour l'échantillon ("C3 \& C4")\\
	Barplot Q100 et Q1000, et comparaison de la distribution a postériori du seuil
	
	\subsection{Quel est l'impact de la méconnaissance du seuil de perception et de son nombre d'années de validité ?}
	Comparaison des résultats du modèle (B) et du modèle (C) pour l'échantillon ("C3 \& C4")\\
	Barplot Q100 et Q1000, comparaison de la distribution a postériori du seuil et de la taille de la période historique\\
	Comparaison de la date de début de la période historique estimée par le modèle (maxpost) avec la date obtenue via la méthode de \cite{prosdocimi_german_2018}.\\

	\subsection{Quel est l'apport des crues historiques pour des durées de chroniques plus courantes ?}
	On dégrade les informations de la période continue pour se mettre dans une situation plus courante. Beaucaire Restitution (1970-2020): période continue, Pont de Beaucaire (1816-1969): période historique.
	\subsubsection{Si le seuil de perception est bien connu}
	Comparaison des résultats du modèle (Baseline) appliqué à Beaucaire Restitution avec les résultats du modèle (A) appliqué à Beaucaire Restitution (continu) et Pont de Beaucaire (occurrences sup-seuil).\\
	Barplot Q100 et Q1000
	
	\subsubsection{Si le seuil de perception et sa durée de validité sont méconnus}
	idem mais avec le modèle (B) et/ou (C)\\
	Barplot Q100 et Q1000, paramètres a posteriori vs vraies valeurs
	
	\subsubsection{Bonus 1}
	Idem au précédent, (modèle (C) et période récente "dégradée") mais en répétant l'expérience pour des durées continues/durée historiques variables. 
	
	\subsection{Bonus 2}
	Si il reste du temps : comparaison "Baseline" 1816-2020 avec (C) et un autre modèle (D) qui prend en compte le débit des crues ayant dépassé le seuil sur la période récente "dégradée".\\
	On juge ainsi de l'intérêt de: connaître tous les débits max annuels VS connaître le débit des max annuels supérieurs à un seuil VS connaître le nombre de dépassements du seuil.
	
\section{Conclusion du chapitre}
	Conclusions sur l'intérêt des crues historique pour l'estimation des quantiles extrêmes à Beaucaire. \\
	Est-ce qu'on observe une réelle amélioration avec les résultats du chapitre 1 ?\\
	Selon les résultats, ajouter quelque chose comme : "l'utilisation des crues historiques peut mener à faire de fortes hypothèses (seuil et durée de la période), il faut être pragmatique sur la considération des incertitudes\\
	Ces conclusions sont valables uniquement à Beaucaire (période continue très longue, paramètre de forme positif, pas de tendance observée due au changement climatique ou autre). Mais que nous à appris l'application du modèle à un échantillon dégradé plus représentatif des longueurs de chronique habituelles ?\\
	Perspectives sur les modèles d'analyse fréquentielle en contexte non-stationnaire et sur les modèles régionaux.  

\printbibliography
\end{document}