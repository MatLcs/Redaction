\documentclass[11pt]{article}
% packages
\usepackage[utf8]{inputenc}
\usepackage{geometry}
\usepackage[pdftex]{graphicx}
\usepackage{tabularx}
\usepackage{dsfont}
\usepackage{multirow}
\usepackage{amsmath,amsfonts,amssymb}
\usepackage{subcaption}
\usepackage{authblk}
%hyperlinks options
\usepackage{hyperref}
\hypersetup{colorlinks=true,linkcolor=blue,filecolor=magenta,urlcolor=cyan,citecolor=cyan}
%bib options
\usepackage[backend=biber,style=authoryear,bibstyle=authoryear,natbib=true,
giveninits=true,uniquename=false,uniquelist=false,% firstinits=false,
maxcitenames=2,date=year, maxbibnames=99,url=false]{biblatex}
\geometry{left=20mm, top=20mm, right=20mm}
%float barrier
\usepackage{placeins}
\addbibresource{Thèse.bib}
\title{Introduction}
\author{Mathieu}


\begin{document}
\maketitle

\tableofcontents

\newpage

\section{Le risque inondation}

	\paragraph{} L'inondation est le type de catastrophe naturelle le plus fréquent dans le monde, mais également celui ayant affecté le plus de personnes au cours des 20 dernières années \citep{undrr_human_2020}. Depuis le début du XXI\textsuperscript{ème} siècle, plus de 100 000 personnes ont perdu la vie dans des inondations à travers le globe. En France, il s'agit du premier risque naturel par l'importance des dommages provoqués et le nombre de communes concernées \citep{medd_site_nodate}. Les inondations peuvent avoir des origines variées : crues, submersions marines, ruissellement, rupture de poche glaciaire, rupture d'ouvrage, etc. Parmi ces différents phénomènes, la crue est le type d'inondation le plus fréquent. 
	
	\paragraph{} Les hydrologues utilisent les chroniques de débit continuellement enregistrées par les stations limnimétriques afin de caractériser statistiquement le risque de crue. Pour cela, ils utilisent le concept de \textbf{période de retour}, qui est également utilisé dans de nombreux domaines liés aux risques naturels. La période de retour est intimement liée à la notion statistique de probabilité au non-dépassement. On peut dire que le débit d'une crue de période de retour $T$ (en années) est en moyenne égalé ou dépassé toutes les $T$ années. On peut également dire qu'un débit de période de retour $T$ a une probabilité $p_1 = 1/T$ d'être dépassé chaque année, ou bien une une probabilité $p_2 = 1-1/T$ de ne pas être dépassé. Il faut noter que ces affirmations ne sont valables qu'à condition que les processus à l'origine des crues soient stationnaires dans le temps. Même si il parait trivial, le concept de période de retour porte souvent a confusion. Par exemple, si la dernière crue centennale de la Seine ($T = 100$ ans) a eu lieu en 1910, cela n'a aucune conséquence sur la probabilité d'observer une crue centennale de la Seine en 2010. Cette probabilité reste en effet égale à $p = 1/100$, que l'on soit en 1910, 2010 ou 2023. A l'inverse, il est tout à fait possible d'observer deux crues centennales deux années de suite. La notion de période de retour est utilisée pour dimensionner des infrastructures ou pour protéger les populations en fonction du risque de crue dans la zone, en tenant compte d'une marge. Par exemple, en France, l'aléa de référence pris en compte dans le Plan de Prévention du Risque Inondation (PPRI) "correspond à un phénomène ayant une probabilité de survenance de 1 chance sur 100 chaque année. S'il existe une crue historique dont la période de retour est supérieure à la crue centennale, cet événement historique est retenu comme aléa de référence". \citep{medd_site_nodate}. Étant donné qu'il y a environ 63\% de chances d'observer au moins une crue centennale en 100 ans, on peut considérer que des infrastructures protégeant les populations jusqu'à la crue centennale auront environ 37\% de chances de couvrir efficacement leur rôle de protection au cours d'une période de 100 ans. La détermination précise du débit correspondant à une période de retour donnée (également appelé "crue de projet" ou "quantile de crue") est donc essentielle.
	
	\paragraph{} A l'origine, l'estimation des quantiles de crues était purement empirique. 
	Détails sur la méthode
	Problèmes de l'approche empirique CF BEN
		
	\paragraph{Approche probabiliste} 

	\paragraph{Modélisation pluie/débit} méthodes stochastiques : GRADEX, AGREGEE, SHADEX, etc




\section{Estimation du risque inondation}
		\subsection{Concept de période de retour}
		Cours Benjamin + Thèse brigode 
		
	    \subsection{Estimations empiriques}
	    Cours Ben + biblio anciennes stedinger etc
	    \subsection{Ajustements statistiques}
	    cours ben + livres FFA : Hamed 2019; Jain 2019
	    ajuster une distribution : block maxima ou Over treshold
	    \subsection{Méthodes régionales}
	    Thèse pierre
	    \subsection{Méthodes pluie/débit}
	    thèse pierre
	    \paragraph{} SHYREG, GRADEX, SHADEX, etc
	    
	    \paragraph{} Quelle que soit la méthode d'estimation du risque retenue, des chroniques de débit sont nécessaires.
	    
\section{Élaboration des chroniques de débit}
		\subsection{Schéma hydrométrique usuel}
		charte de l'hydrométrie et refs internationnales (article JOH)
		\subsection{Quantification des incertitudes}
		uH, u jau, u RC, détarages, SPD
		
		\subsection{Hydrométrie en contexte historique}
		incertitude sur les débits de pointe histo : pas d'échelle\\
		difficulté d'avoir un échantillon exhaustif\\
		reconstitutions par la modélisation\\
	
		
\section{Analyse fréquentielle des crues}
		\subsection{Méthodes d'échantillonnage}
			\subsubsection{Max annuels}
			\subsubsection{Sup-seuil}
			
			ATTENTION AUX DOUBLONS AVEC SECTION 2
		\subsection{Sources d'incertitude}
			\subsubsection{Chroniques de débit}
			\subsubsection{Échantillonnage}
			\subsubsection{Choix de distribution}
			\subsubsection{Respect de l'hypothèse de stationnarité}
			
Sharma et al, 2018 : Despite evidence of increasing precipitation extremes, corresponding evidence for increases in flooding remains elusive. 
		
\section{Analyse fréquentielle des crues en contexte historique}
		\subsection{Intérêt de l'utilisation des données historiques}
		
		Illustration avec l'exemple de l'Ahr en 2021 : Ludwig et al, 2023 -> Figure 7
		"Again, this underlines the challenges of extreme value
statistics and the large uncertainties when estimating return
periods for the 2021 event. It also indicates the need for
even longer historical time series and reconstructions as far
as possible and/or the examination of the completeness of
the events between 1804 and 1946 as well as before 1804,
where there is evidence that over 70 floods occurred in the
Ahr river basin since the year 1500, including the large 1601
event (Seel, 1983). In addition, 1818 and 1848 were also
large events with currently no reconstructed streamflows."
		
		\subsection{Données continues anciennes}
		Nombreuses données qui dorment dans les archives : illustrations ? Données dispo en banque hydro VS réalité
		\subsection{Données pré-enregistrements continus}
		Liste d'études histo dans le monde, en europe et en France
		\subsection{Incertitudes autour de l'analyse fréquentielle historique}
		
\section{Le risque inondation dans la basse vallée  du Rhône}

\section{Organisation du manuscrit}






\printbibliography
\end{document}
