\documentclass[a4paper,11pt]{article}
\usepackage{longtable} % for 'longtable' environment
\usepackage{lscape} % for 'landscape' environment
\usepackage{geometry} 
\geometry{a4paper}

%\geometry{margin=2.5cm}
%\geometry{headheight=6mm}

% Si on a besoin d'une configuration plus précise des marges
\geometry{%
  a4paper,                % format de papier
% Définition des marges :
  left= 1cm,            % marge intérieure à la page
  right = 1cm,          % marge extérieure
  top = 0.5cm,
  bottom = 0cm,
% En-tête et pied de page :
%  headheight=6mm,         % espace réservé à l'en-tête dans la marge top
  %headsep=3mm,            % espace entre le corps et l'en-tête
  %footskip=9mm            % espace entre le corps et le pied de page
}

\begin{document}

%\newgeometry{left=0.8cm,right=0.8cm,top=1.5cm,bottom=1.5cm}

\begin{landscape}
  
	\begin{tiny}  
  
	\begin{longtable}{|l|p{1cm}|
	p{1.5cm}|p{2cm}|p{2.5cm}|p{1.5cm}|p{1.5cm}|p{2cm}|p{2cm}|p{2cm}|p{2cm}|p{2cm}|p{2.5cm}|p{2cm}|}
	
    \hline
  \textbf{Année} & \textbf{Mois} & \textbf{Modélisable } & \textbf{Description} & \textbf{Causes} & \textbf{Info affluents} & \textbf{Hauteurs échelle} & \textbf{Info hauteur Beaucaire - Tarascon} & \textbf{Info hauteur Arles} & \textbf{Info hauteur Avignon} & \textbf{Indices hauteur autre} & \textbf{Ouverture digues} & \textbf{Autres infos} \\ \hline
  \endhead
        \textbf{1353} & mai & Non & Le Rhône inonde la plaine depuis Avignon jusqu’à la mer. & Gel du Rhône et de la Durance & ~ & ~ & ~ & ~ & ~ & x & ~ & ~ \\ \hline
        \textbf{1398} & octobre & Non & Crue dévastatrice du Rhône selon les chroniques d'Avignon et de l'arlésien Bertran Boysset & ~ & ~ & ~ & ~ & ~ & ~ & inférieur de 10 cm à la crue de 1396 à Arles (qui est une C3?...) & ~ & ~ \\ \hline
        \textbf{1433} & novembre & Oui (Avignon) & Grande inondation qui s'étend d'Avignon à la mer & Pluie et fonte des neiges (en novembre?) & Débordement conjoint Durance et Sorgue & 6,77 m au dessus de l'étiage à Avignon, échelle indéterminée et 7,08 m au pont suspendu d'Avignon & ~ & ~ & Miracle chapelle des pénitents gris, 1 m d'eau dans la nef & ~ & ~ & Toute la Carmarque est inondée, des pirates y pénètrent et se livrent à des pillages \\ \hline
        \textbf{1471} & septembre & Non & L'une des plus grandes inondations conjuguées du Rhône et de la Durance & Probablement cévenol, caractère brutal & Durance en crue & X & ~ & ~ & une partie des murailles abattues, repère de crue église st michel de Caderousse, 1m52 au dessus du dallage & Caderousse, chapelle d'Ancezune, 1m52 au dessus du dallage & à Arles & ~ \\ \hline
        \textbf{1529} & novembre & Oui & "Grande inondation du Rhône dit de Saint-Martin (""La Ronada de San Martin"") à Arles." & X & X & 5,25 sur l'échelle de Véran (Arles) & porte St Jean de Tarascon endommagée par la crue & Trébon, plan du bourg et camarque inondés & 300 m de remparts abattus & ~ & Levées de la chaussée du trébon endommagée (arles) & ~ \\ \hline
        \textbf{1548} & novembre & Oui (Avignon) & Caractère foudroyant à Avignon mais peu d'échos à Beaucaire et Arles & crue à prépondérance océanique très nette, mais avec des apports duranciens et cévenols évoquant aussi des pluies méditerranéennes, limitées à un secteur étroit autour du couloir rhodanien & Débordement de la Durance à Avignon & X & Chaussées ruinées à Tarascon, murailles démolies & Hôpital St Lazare endommagé & " L'eau entre dans la ville ""jusques à la Saunerie, à Saint-Agricol, à la Croix de Lunel, à Sainte-Catherine … HR.   L'eau atteint la coquille de la chapelle St Nicolas sur le pont St benezet,   25 cm en dessous du plus haut de la porte de la Ligne,   tient toutes les arcades du pont et à environ 1m50 avant de toucher le plus haut des arcs. RAPPORT COMPLET à revoir sur des contradictions concernant les marques de crues à Avignon" & nombreux indices contradictoires à Avignon quand au dépassement par cette crue de celle de 1856. & ~ & ~ \\ \hline
        \textbf{1570} & décembre & Hauteur seulement & Inondation générale du Rhône, concernant l'ensemble du bassin. Inondation extraordinaire commençant à Lyon le 2 décembre 1570 et qui atteint Avignon le 5 décembre, et emporte ensuite les chaussées d'Arles. & Pluies océaniques + fonte des neiges suite à un réchauffement brutal & X & 5,17 sur l'échelle de Véran à Arles & Le Rhône déborda de telle sorte qu'il tenoit toute la Camargue, les plans de St Gilles, Bellegarde, jusques à Beaucaire Tresbons et le Plan du Bourg & Chaussées emportées & Quartiers bas de la ville inondés & ~ & ~ & Lyon inondé, maisons détruites à la Guillotière \\ \hline
        \textbf{1573} & octobre & Oui & "Inondation ""extraordinaire"" du Rhône à laquelle on peut attribuer l'ouverture du Grau du Roi à Aigues-Mortes" & ~ & X & X & L'eau passe par-dessus les chaussées,principalemnt à la pauze St Martin. Tout le plat pays et terroire de Beaucaire est en eau, jusqu'aux abords de la montagne.  & X & Deux arches du pont d'avignon emportées & moins haute que celle de 1548 de 4 doigts à Caderousse. Tout Caderousse est sous l'eau sauf vers l'église et la place & "Vers Beaucaire, plus de 6km de chaussées ruinées. Dont 25 cannes à la Bergantière, 500 cannes au dessus du port (de St Gilles)  une autre au trou du Sauze. à St Gilles, la crue ne cause pas tant de tord qu'à Beaucaire car le torrent de Rebayres nettoie la grande robine qu'ils tiennent ouvert afin de vidanger le terroir. Les travaux sont urgents car le Rhône menace d'inonder l'ensemble du terroir jusqu'à l'étang de Maugio. à tarascon :  Grande chaussée endommagée et rompue en plus de 10 ou 12 endroits ""tant dessus que dessous de la présente ville""" & Ouverture du Grau du Roi \\ \hline
        \textbf{1580} & août & Oui (Avignon) & "Le ""déluge"" n'aurait duré qu'un jour, soit il fut d'une intensité hors de toute norme, soit il intervenait sur des sols déjà détrempés, mais la période ne penche guère vers cette hypothèse. On doit donc davantage penser à une crue cévenole." & Pluies épouvantables du 24 août (Selon M Villard) & La Durance resta calme & X & ~ & Un ane trouvé sur le toit d'une maison ( mas de l'Ase) & L'eau dépasse et renverse le mur que soutient la coquille de la chapelle (Saint-Nicolas) du pont Saint-Bénezet. 6 pans d'eau à la place devant la croix. 60 maisons tombées à Caderosuse. L'eau rompit une porte du portal St Lazare & mur de soutien de la chapelle du pont d'avignon renversé & Beaucaire : Chaussées démolies en 4 endroits : les Saussan, au Radeau et deux endroicts du Coussac, ainsi qu'à la pauze St martin. A Arles : Tresbon , plan du bourg, corrèges et montlong, grande quantité d'ouvertures aux chaussées. & ~ \\ \hline
        \textbf{1674} & novembre & Oui, QUID de la Durance qui se jette en partie dans le Rhône au niveau d'Arles? & La plus grave inondation rhodanienne survenue au XVIIe siècle avec important apport de la Durance. Toute la basse Provence et tout le bas Languedoc sont touchés. & 15 jours de pluie abondante et fonte des neiges (déjà?) & Très grande crue de la Durance les 15 et 16 novembre avec débordement à Avignon. Les eaux de la Durance se jettent en partie dans le Trébon par le débouché de Saint-Gabrieln (près de Tarascon). Débordement dévastateur de la Sorgue le 15 novembre (Sorgues), Drac et Romanche en crue, donc isère ? & 6,55 Pt susp Avignon, 6,38 Madone Avignon, 5,39 à Arles (Véran) & 1,50 m d'eau dans tout Tarascon, pont beaucaire tarascon emporté, digues brisées, chaussées détruites… 8 ou 9 pans d'eau devant la maison de ville de Tarascon & 2m dans l'église St Lazare d'Arles, 3m sur le pont de Crau &  de l'eau jusqu'à la barbe de la statue de saint françois (?) à Avignon, 2m40 dans le cloitre des minimes, 0,80 au dessous de la crue de 1755. Repère placé rue st michel sur l'église des célestins. Porte St lazare enfoncée & ~ & Chaussées du Trébon, plan du bourg, Roque de l'acier près de Tarascon, Lansac, et de l'isle de Camargue emportées. Toutes les chaussées de boulbon jusquà Tarascon renversées en raison de l'avulsion de la Durance vers Arles par le Vigueirat. Nombreuses ruptures à Beaucaires(toutes décrites dans la transcription) & Fort transport solide qui comble en partie le lit de la Durance, d'où son avulsion \\ \hline
        \textbf{1694} & novembre & Oui & "Les eaux surmontent les chaussées et se répandent avec une rapidité prodigieuse dans tout le terroir depuis Saint-Gabriel à la mer : ""on allait par bateaux de Tarascon jusques à la mer, que l'isle de Camargues était également couverte d'eau et qu'on allait aussi d'Arles à Maguelone, en Languedoc, par bateaux à travers les champs.""" & pluies soudaines et vent de mer & Débordement de la Durance à Avignon & 4,95m échelle Véran d'Arles & 3 pieds d'eau dans le jardin des capucins de Tarascon mais pas d'eau dans le bas du couvant. Plan du bourg et trébon inondés. L'eau s'étend de la ville au ténément de Beynes.  & Le rhône entre dans la ville par la porte de Rousset et autres endroits le long du quai. Chaussée de Fourque rompue, l'eau se répand violemment dans toute la Camargue & Crue égale à 1679 à Avignon & 4,95 m Véran (arles) & Arles : 870 cannes de chaussées ouvertes [1 740 mètres] : au-devant du port de Fourques, au Clot Négadier et le long du grand Rhône (Rougnouse, Montlong, ...) et une quinzaine d'autres endroits (au-dessous d'Albaron, ...), ainsi que les chaussées des terres de la Porcelette, du Grand Passon et de Gouine, d'environ100 mètres. Beaucaire : 400 m de chaussés à la Lèque & beaucoup de sable partout sur le territoire à partir de beaucaire. Ponts de bateaux de Beaucaire et Arles rompus. Le vent de mer ramène des vagues jusque 3 lieues dans la terre ferme. On va d'Arles à Montpellier en bateau à travers champs \\ \hline
        \textbf{1705} & novembre et Janvier suivant & Oui & "Le Rhône devient ""furieusement gros"" suite aux pluies." & une semaine de pluie continue & Durance et Verdon en crue & 5,03 à Arles (Véran) & ~ & 65 cm plus haut que 1647, de l'eau 50 cm au dessus de la grande chausée du Trébon, 2m d'eau sur le chemin du pont de Crau, 1,50m dans le quartier du Plan de Bourg, 1,3m partout depuis le pont de Crau jusqu'au bois des Cays, L'eau de la durance ayant déversé depuis chateaurenard arrive à arles via les terroirs.  & plus  d'1,25m d'eau dans les terres de Barbantane & 5,03 m Véran (Arles) & 300 m de chaussées détruites sur le terroir de Tarascon (infos plutôt précises sur la hauteur et largeur de chaque ouverture dans la transcription). Ouvertures faites lors de la première inondations élargies lors de la seconde.   & sable et limon dans les plaines \\ \hline
        \textbf{1706} & Janvier & Oui, commun avec la précédente & "La récurrence de janvier 1706 achève de ruiner les pays riverains du Rhône et de la Durance. Inondation généralisée : ""les terroirs ressemblent à une mer. """ & pluies abondantes et prolongées & "Débordement de toutes les rivières en haute et basse Provence (Bassins de la Durance, du Verdon, de la Bléone, de l'Asse et des bassins côtiers ) et une quantité phénoménale de ""vallons"", ruisseaux et torrents." & ~ & ~ & Partout entre 2,3 et 2,6 m d'eau, 1,6m au dessus du pont de crau et 75cm dans le jardin des pénitentes St Genest, Trébon et plan du bourg inondés & marque à la porte St lazarre & 5,03 m Véran (arles),  dépasse de 3 ou 4 pieds celle de novembre à avignon, 5 pieds au dessus du pont de Crau à Arles & Tarascon : deux ouvertures de 8 cannes de long, fort profondes, le rhône a passé presque par-dessus toutes les chaussées. Beaucaire : 5 Cannes 4 pans devant la terre de la comanderie St Pierre, 15 cannes juste à côté, 8 Cannes, 9 et 2 Cannes, une canne, 14 cannes, deux cannes. & idem \\ \hline
        \textbf{1711} & février & Oui & "Cette grande crue ne semble pas avoir été causée par des pluies méditerranéenne mais par la conjonction de pluies océaniques(""du côté de Lion"") et des fontes de neige (?) [à la suite de ces pluies ?]. Les Mémoires de Louis Pic confirment ces pluies océaniques qui se déversèrent un mois sur la Savoie, la Bourgogne, le Lyonnais et le Dauphiné." & Pluies océaniques et fonte des neiges en savoie, bourgogne, lyonnais et dauphiné & ~ & ~ & digues brisées à beaucaires, pas vu depuis 1674 & 1,5 à 1,75m sur le pont de Crau, 1711 moins haute de 27cm que 1706, d'après des marques à la maison des repentis (St Genest) . Et d'autres assurent que la crue est plus grosse qu'en novembre 1705 & ~ & entre 1,5 et 1,75m sur le pont de Crau à Arles, et 10 pouces de moins qu'en 1706 à Saint Genêt, supérieur à celle de 1706 à avignon : marques sur la porte saint-lazare & Petit Rhone : Deux ruptures à la chaussée : 72 toises [environ 140 mètres] près de Casenove; 30 toises [60 mètres] à la distance de deux lieues en aval, Arles : Chaussées endommagées entre Fumemorte et Trinquetaille sur 6 lieues, 1km de ruptures vers fumemorte, 100 m à plan du bourg, brèches vers la mourade de Blanc, Boulbon chausséees emportées, Chaussée du trébon emportée depuis la porte St jean jusqu'aux limites d'arles sur environ 1,3km, 2,4km au quartier de la contamine à tarascon & idem \\ \hline
        \textbf{1745} & novembre & Oui & Inondation pluviale puis débordement du Rhône avec 3 montées successives des eaux : du 4 au 6 (maxima), du 12 au 14, puis les 16 et 17 novembre. Décalage de la crue entre Avignon et Arles qui peut s'expliquer par l'hypothèse suivante : le flot durancien se contenta d'abord de grossir le débit du Rhône en aval du confluent. Puis, l'effet d'obstacle des eaux de la Durance eut le temps de se faire sentir à l'amont immédiat, c'est-à-dire à Avignon. & Pluies abondantes et continuelles depuis 1 [ou 2 mois selon les sources] qui redoublent du 4 au 7 novembre. Nouvelle vague de pluie du 12 au 14, et les 16 et 17 novembre, Crue initiée par la Durance puis le Rhone, effet d'obstacle de la Durance sur le Rhone à avignon. & Trois crues sucessives de la Durance également avec dégâts à Mallemort et Pertuis. Le 4-5 novembre, crue du Gardon. & 5,31 à Arles (Véran) & 2,5 m d'eau dans les maisons de Tarascon,  Tarascon est en partie sous 10 pieds [3,25 m] d'eau et il y a 10 à 12 pieds d'eau dans la plupart des maisons & 3,5m sur le chemin du pont de Crau & 16 cm d'eau à la Porte de l'Oule. Tous les quartiers bas de la ville inondés : Careterie, Pénitents Gris, corps saints, minimes, recolets, capucins et dominicains.  & ~ & Brèche de 60m de large à la chaussée du Trébon à 500m d'Arles. 600m d'ouvertures au pas du bouquet (Tarascon). Rupture de la roubine du Vigueyrat (Arles). Chaussée de Camargue rompue en plusieurs endroits du côté du Baron.  & idem. Pont de Bateaux Beaucaire-Tarascon emporté, muraille du pont de Lansac (Tarascon) détruite, pont de Bateaux d'Arles détruit par celui de Beaucaire, à avignons, le rhone a diminué de 10 pieds d'un coup certainement suite aux ruptures de digues à Tarascon.  \\ \hline
        \textbf{1755} & décembre & Oui & La plus importante crue du XVIIIème siècle. Maximum dans la nuit du 30 novembre au 1er décembre. Longue stagnation de l'eau dans les terroirs toujours sous l'eau fin décembre. & "Précédant la crue : longue période de vent humide ou vent marin d'Est qui souffla ""huit à dix jours"" et qui déversa ses masses d'eau sur les Cévennes et le Vivarais." & Grand débordement de la Durance qui inonde la plaine de Barbentane, ainsi que le Trébon et les marais d'Arles (par la gorge de Saint-Gabriel). Le 3 décembre, inondation de l'Ouvèze à Bédarrides. & 5,47m et 5,88m à Arles (Véran et Rhonomètre), 7,23m et 7,5m à Avignon (Madone et Pont suspendu) & 2,6m d'eau dans la plaine de Beaucaire, 2m d'eau dans les maisons du quartier notre dame de bonne aventure de Tarascon, 2,75m dans les maisons à coté de la porte St Jean, 2,6m d'eau dans le terroir, montent jusqu'au premier étage dans la ville basse. 1,75m dans l'église des Capucins. & 33cm plus haut que 1745 à Aigues mortes, 1m dans les quartiers du Trébon et plan du Bourg, 1,75m dans l'église des pp Recolets et 1,25m dans celle des augustins reformés. De 2 à 4m d'eau dans le terroir de Fourques. 0,8m plus haut que 1711 et 1747 à Pont St Esprit & 3m24 dans les maisons des faubourgs d'Aramon, 2,5m d'eau dans les bas quartiers d'Avignon, de l'eau au dessus de la tête de St François (sur le pont), 0,65m dans l'église St didier, 3,3m contre les murailles de la ville, de la porte de l'oule jusqu'à la porte St roch, 3,75m dans le jardin des ursulines, 2,5m dans l'église des recolets, minimes, St André, Carmélites, pénitents gris et cordieliers. Nombreuses marques existantes & beaucaire : 2,6 m d'eau dans la plaine, tarascon, 2,6 m d'eau dans les maisons, 2,75m dans les maisons à côté de la porte de st jean, 1,75m dans l'église des capucins, avignon : 7,23 à l'échelle madone, 5,44 à l'échelle véran d'arles & brèche à la chaussée d'Argence, les eaux remontent jusqu'à environ 12 km dans le terroir & idem \\ \hline
        \textbf{1801} & novembre & Oui & "Crue méditerranéenne extensive selon M. Pardé. Inondation par les chaussées ouvertes à Tarascon depuis le 11 octobre, puis rupture des chaussées inférieures, à l'aval. De Boulbon à la mer est décrite ""une seule étendue d'eau""." & Grandes pluies d'automne (environ 570 mm à Arles du 17 septembre au 5 décembre ; 316,5 mm à Marseille en novembre) & Crues de l'Ardèche, du Gardon et de l'Ouvèze. Crue simultanée de la Durance qui surmonte ses chaussées entre Barbentane et Châteaurenard et atteint 5 m à Mirabeau et 3,42 m à Bonpas les 11-12 novembre & 5,11 à Arles (Véran), 6,94 et 7,2 à Avignon(Madone et Pont Suspendu) & 5,5cm au dessus de 1755 à Tarascon & De l'eau jusqu'au 1er étage rue de la cavalerie, Trébon, Plan du Bourg et Camargue sous plus de 3m d'eau, 2,35m au dessous du trottoir du pont de Crau, 60 cm d'eau dans l'écurie du mas du Radeau (plan du bourg), 3m d'eau dans le Trébon & hauteur inférieure à 1755 de 30cm. Repère existant sous le rochers des doms, porte du rhônne & tarascon : 5,5cm au dessus de 1755, avignon : 6,92 à l'échelle de la madone et 7,2 au pont suspendu, arles : 5,11 sur l'échelle de Véran & Plan du Bourg(RG gd rhône) : Chaussée renversée à côté de la roubine de meyranne, plusieurs brèches à la chaussée du Trébon, à la porte de la cavalerie d'Arles), au plan du bourg inférieur. Tarascon : Digue amont de Tarascon au pas du bouquet complètement renversée, pont acqueduc de crau endommagé. & fortes disparités sur le dépassement de la crue de 1755 selon les localités (plus bas à Avignon et Arles, plus haut à Beaucaire. (P 2 et 3 transcription) : 'l'ouveze et l'ardèche avoient le plus fourni, que la première surtout avoit changé de lit, que la Durance avoit été débordée, mais qu'elle n'avoit pas donné la même quantité d'eau qu'en 1755, qui s'étoit jointe avec le Rhone dans le territoire d'avignon, ce qu'lle n'avoit pas fait dans cette dernière circonstance, ensuite que la campagne d'Avignon avoit été moins inondée qu'en 1755' \\ \hline
        \textbf{1810} & mai & Oui & "Crue de printemps à comparer avec celle du 31 mai 1856. Le Rhône est ""plein"" dès le 20 mai et inonde du 24 mai au 1er juin puis lente décrue. Presque toutes les rivières de l'Europe ont débordées. Depuis Lyon jusques à la mer, les inondations ont fait des dégâts étonnants." & Abondance des pluies jointe à la chaleurs des vents méridionaux, et dégel du Rhône & Crue de l'Ardèche, du Guil, … & 5,2m à Arles (Véran) et 5,93m à Avignon (Madone) & 10cm au dessus de 1755, 2,18m au dessous de 1856 (???) : repère qui se trouve presque à l'extrémité amont de la rampe d'accès du port d'amont. & 5,38 à+/- 0,135m. Rhône à la hauteur de la fleur de lys (? Hôtel de ville ?)  & sur le point d'inonder le bas quartier de la ville & ~ & brêche de 184m à Barbe d'Ase (Gd Rhône), 8m d'eau en profondeur ? - Plusieurs ouvertures vers le bas Plan du bourg : territoires de parade et passon - Petit Rhône : corrège, 3 brèches entre les passerons et la martelière de la cape dont une d'au moins 50m, 5 brèches de 20 à 30m au petit plan du bourg, entre mas de la ville et montcalde - Trois brèches entre tarascon et lansac sur la chaussée du Trébon à Tarascon, trois brèches au dessus de Tarascon. Brèche de 200m à la crois de seignoret (Arles), une brèche de 184m à la chaussée de montlong & Du jamais vu de mémoire d'homme pour un mois de mai \\ \hline
        
	\end{longtable}

\end{tiny}


\end{landscape}


\end{document}