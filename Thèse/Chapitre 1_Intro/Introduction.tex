\documentclass[11pt]{article}
% packages
\usepackage[utf8]{inputenc}
\usepackage{geometry}
\usepackage[pdftex]{graphicx}
\usepackage{tabularx}
\usepackage{dsfont}
\usepackage{multirow}
\usepackage{amsmath,amsfonts,amssymb}
\usepackage{subcaption}
\usepackage{authblk}
%hyperlinks options
\usepackage{hyperref}
\hypersetup{colorlinks=true,linkcolor=blue,filecolor=magenta,urlcolor=cyan,citecolor=cyan}
%bib options
\usepackage[backend=biber,style=authoryear,bibstyle=authoryear,natbib=true,
giveninits=true,uniquename=false,uniquelist=false,% firstinits=false,
maxcitenames=2,date=year, maxbibnames=99,url=false]{biblatex}
\geometry{left=20mm, top=20mm, right=20mm}
%float barrier
\usepackage{placeins}
\addbibresource{Thèse.bib}
\title{Introduction}
\author{Mathieu}


\begin{document}
\maketitle

\tableofcontents

\newpage

\section{L'estimation du risque inondation}

	\paragraph{} L'inondation est le type de catastrophe naturelle le plus fréquent dans le monde, mais également celui ayant affecté le plus de personnes au cours des 20 dernières années \citep{undrr_human_2020}. Depuis le début du XXI\textsuperscript{ème} siècle, plus de 100 000 personnes ont perdu la vie dans des inondations à travers le globe. En France, il s'agit du premier risque naturel par l'importance des dommages provoqués et le nombre de communes concernées \citep{medd_site_nodate}. Les inondations peuvent avoir des origines variées : crues, submersions marines, ruissellement, rupture de poche glaciaire, rupture d'ouvrage, etc. Parmi ces différents phénomènes, la crue est le type d'inondation le plus fréquent. 
	
	\paragraph{} Les hydrologues utilisent les chroniques de débit enregistrées en continu aux stations limnimétriques afin de caractériser statistiquement le risque de crue. Pour cela, ils utilisent le concept de "période de retour", qui est également utilisé dans de nombreux domaines liés aux risques naturels. La période de retour est intimement liée à la notion statistique de probabilité au non-dépassement : on peut dire que le débit d'une crue de période de retour $T$ (en années) est en moyenne égalé ou dépassé toutes les $T$ années. On peut également dire qu'un débit de période de retour $T$ a une probabilité $p_1 = 1/T$ d'être dépassé chaque année, ou bien une probabilité $p_2 = 1-1/T$ de ne pas être dépassé. Il faut noter que ces affirmations ne sont valables qu'à condition que les processus à l'origine des crues soient stationnaires dans le temps. Même si il parait trivial, le concept de période de retour porte souvent a confusion. Par exemple, si la dernière crue centennale de la Seine ($T = 100$ ans) a eu lieu en 1910, cela n'a aucune conséquence sur la probabilité d'observer une crue centennale de la Seine en 2010. Cette probabilité reste en effet égale à $p = 1/100$, que l'on soit en 1910, 2010 ou 2023. A l'inverse, il est tout à fait possible d'observer deux crues centennales deux années de suite. La notion de période de retour est utilisée pour dimensionner des infrastructures ou pour protéger les populations en fonction du risque de crue dans la zone, en tenant compte d'une marge. Par exemple, en France, l'aléa de référence pris en compte dans le Plan de Prévention du Risque Inondation (PPRI) "correspond à un phénomène ayant une probabilité de survenance de 1 chance sur 100 chaque année. S'il existe une crue historique dont la période de retour est supérieure à la crue centennale, cet événement historique est retenu comme aléa de référence". \citep{medd_site_nodate}. Étant donné qu'il y a environ 63\% de chances d'observer au moins une crue centennale en 100 ans, on peut considérer que des infrastructures protégeant les populations jusqu'à la crue centennale auront environ 37\% de chances de couvrir efficacement leur rôle de protection au cours d'une période de 100 ans. La détermination précise du débit correspondant à une période de retour donnée (également appelé "crue de projet" ou "quantile de crue") est donc essentielle.
	
	\paragraph{} A l'origine, l'estimation des quantiles de crues était purement empirique. Il s'agissait par exemple, pour une station de mesure donnée, de calculer la fréquence cumulée des débits maximum annuels, classés par ordre croissant. On pouvait alors accéder à de premières estimations de la probabilité au non-dépassement pour un débit donné. Cependant, cette approche comporte de nombreuses limitations, notamment quand il s'agit d'extrapoler au delà du plus fort débit connu. Désormais, la pratique courante, appelée analyse fréquentielle des crues, consiste à estimer les paramètres d'une distribution statistique (préalablement choisie selon la variable hydrologique étudiée) en se basant sur les observations. Cette pratique comporte l'avantage, par rapport aux estimations empiriques, d'être moins sensible à la taille de l'échantillon d'observations disponible. De plus, elle offre la possibilité d'extrapoler au delà de la plus forte crue connue, ce qui permet d'accéder à de grandes périodes de retour qui sont nécessaires pour le dimensionnement d'infrastructures (par exemple jusqu'à $T =$ 10 000 ans pour les barrages français \citep{le_delliou_recommandations_2014}). 
	
	\paragraph{} Une autre famille de méthodes probabilistes d'estimation des crues se base sur l'utilisation de modèles hydrologiques (ou modèles pluie/débit) qui simulent la transformation de la pluie en débit au sein de bassin versant. Il s'agit, dans le cas de l'estimation du risque de crue, de simuler une pluie extrême dans un contexte de saturation en eau du bassin versant. Sous cette hypothèse, la distribution des débits est conditionnée à la distribution des pluie extrêmes. Il n'existe pas de consensus scientifique en faveur des méthodes classiques d'analyse fréquentielle (basées uniquement les chroniques de débit), ou en faveur des méthodes pluie/débit. La présente thèse s'intéresse uniquement aux méthodes d'analyse fréquentielle classiques.
		
		
	\section{Analyse fréquentielle des crues et incertitudes}
	
	\paragraph{} L'analyse fréquentielle des crues, bien que très largement utilisée dans le monde, est affectée par plusieurs sources d'incertitudes qui sont bien souvent négligées. De nombreuses décisions découlent des résultats de l'analyse fréquentielle : dimensionnement des digues de protection pour les populations et les infrastructures à risque, plans d'urbanisme, dimensionnement des évacuateurs de crue des barrages, arrêtés de catastrophe naturelle, etc. Une estimation complète des incertitudes qui affectent cet exercice est donc indispensable afin d'appréhender correctement l'étendue du risque de crue. Ces incertitudes peuvent être divisées en quatre catégories :
	\begin{itemize}
	
		\item Les incertitudes hydrométriques, qui affectent les données de débit, proviennent de la complexité d'estimer en continu le débit d'un cours d'eau en un point donné.
		
		\item L'incertitude d'échantillonnage, qui provient de la longueur limitée de l'échantillon de données disponible.

		\item Les hypothèses de modélisation telles que le choix d'une distribution statistique adaptée à la variable hydrologique étudiée, ou l'hypothèse de stationnarité, qui garantit que les données utilisées sont des représentations d'une variable aléatoire indépendante et identiquement distribuée (ou iid).
	\end{itemize}
	
	\subsection{Incertitudes hydrométriques}
	
	\paragraph{} L'analyse fréquentielle des crues se base généralement sur des données de débit estimées au droit des stations hydrométriques. Le débit des cours d'eau naturels ne peut malheureusement pas être mesuré en continu. En revanche, il est possible de mesurer continuellement la hauteur d'eau en un point donné à l'aide d'une échelle limnimétrique installée à demeure. De plus, des estimations ponctuelles du débit peuvent être réalisées via diverses méthodes de mesure appelées "jaugeages". Sous certaines conditions, il est possible d'établir une relation univoque entre la hauteur d'eau et le débit en un point donné à l'aide des jaugeages. Cette relation nommée "courbe de tarage", constitue le cœur de l'hydrométrie. Chacune des étapes du schéma hydrométrique décrit ci-dessus est affectée par des incertitudes, qui entrainent une incertitude autour des débits estimés (\citet{mcmillan_benchmarking_2012}, \citet{puechberty_charte_2017}). 
	
	\paragraph{} Tout d'abord, plusieurs sources d'incertitudes autour de la mesure de la hauteur d'eau sont identifiées dans la littérature (\citet{van_der_made_determination_1982}; \citet{petersen-overleir_uncertainty_2005}; \citet{mcmillan_benchmarking_2012}; \citet{horner_impact_2018}). Elles concernent notamment la précision de la lecture visuelle de l'échelle limnimétrique, et dans le cas de mesures automatisées, la précision des capteurs et la calibration de ces derniers. La fréquence des relevés entraine également des erreurs d'interpolation, particulièrement dans le cas de chronique anciennes pour lesquelles les relevés étaient effectués visuellement par un opérateur, et étaient donc moins fréquents qu'avec les systèmes automatiques modernes. Cependant, ce type d'erreur n'est que très rarement abordé dans la littérature, alors qu'il peut être particulièrement impactant dans le cas de relevés anciens (\citet{hamilton_quantifying_2012}; \citet{kuentz_hydrometrie_2014}).
	
	\paragraph{} Les courbes de tarage représentent une des plus importantes sources d'incertitude en hydrométrie. Les jaugeages, données indispensables à l'élaboration des courbes de tarage, sont eux-même impactés par des incertitudes qui dépendent de la méthode de mesure \citep{lecoz_quantification_2014}. De plus, la réalisation de jaugeages est particulièrement complexe en situation de crue. Le processus d'estimation de la courbe de tarage est également affecté d'incertitudes, provenant d'une part du modèle choisi pour représenter les conditions hydrauliques du cours d'eau, et d'autre part de l'estimation des paramètres de ce modèle. L'estimation de l'incertitude des courbes de tarage est très largement étudiée dans la littérature (\citet{petersen-overleir_bayesian_2009}; \citet{juston_rating_2014}; \citet{le_coz_combining_2014}; \citet{morlot_dynamic_2014}; \citet{coxon_novel_2015}; \citet{mcmillan_rating_2015}; \citet{mansanarez_rapid_2019}). Il faut également noter qu'une courbe de tarage a une validité temporelle limitée. En effet, la relation hauteur/débit est susceptible de varier dans le temps au gré des changements morphologiques causés par les crues, des travaux dans le lit mineur, de la croissance de la végétation aquatique, etc. Ainsi, la précision des séries de débit est dépendante du contrôle fréquent de la relation hauteur/débit via la réalisation de jaugeages. Les ruptures temporelles de cette relation se nomment "détarages". Leur détection et leur impact sur l'incertitude des séries de débit constitue un sujet particulièrement étudié dans la littérature (\citet{westerberg_stage-discharge_2011}; \citet{guerrero_temporal_2012}; \citet{morlot_dynamic_2014}; \citet{lapuszek_methods_2015}; \citet{mcmillan_impacts_2010}; \citet{darienzo_detection_2021}; \citet{mansanarez_shift_2019}). Lorsqu'on s'intéresse particulièrement aux crues 
	
	en france la plupart des stations ne sont pas jaugées sous Q2 
	
	liste et limitations dans la littérature
	extrapolation CT (lang 2010)
	
	\subsection{Incertitude d'échantillonnage}
	taille chronique kjeldsen et autres
		\subsubsection{Analyse régionale}	
		
		\subsubsection{•}
	
	\subsection{Hypothèses de modélisation}
	distribution, max an ou sup-seuil, hypothèse de stationnarité (climat, anthropique, autre)
	
	description rapide, pas abordé dans la thèse.
	
	

		

%1/ poser le problème (opérationnel) et montrer qu'il est important de le résoudre : c'est donc très bien de démarrer sur le risque inondation, en allant rapidement à l'importance de disposer de stats fiables et précises sur les crues (importance de l'analyse fréquentielle des crues, qui est pourtant largement foireuse et mal faite, comme chacun sait!!). J'imagine que la réglementation en vigueur est également largement débile et ignorant les niveaux réels d'incertitude, comme d'hab... Tu peux parler prévention des inondations mais aussi protection / sûreté (dimensionnement des ouvrages basés sur Q100, Q100000000, etc.) et prise de décision (ex : arrêtés catnat basés sur Q10). Je pense que ton §2 (et aussi le §3) est en partie à regrouper ici car tu as besoin d'expliquer la méthodo de base.
%
%2/ montrer que les méthodes actuelles sont limitées : réseaux hydro limités en nb, espace et temps (cf. chiffres dans mon HDR), hypothèses des analyses stats (stationnarité, distrib...), etc. Là, tu peux citer les travaux qui ont cherché à dépasser ces limites, et indiquant tous les pbs qui restent non résolus. Et notamment l'utilisation de données historiques, LA piste prometteuse, mais en listant ce qui n'a pas été traité ou pas bien (incertitudes hydrométrie, méthodo intégrée de bout en bout, etc.).
%
%3/ expliquer ce que tu vas faire (objectifs et démarche scientifiques) : objectifs généraux, le site d'étude et pourquoi lui, l'organisation du manuscrit (il faut qu'on retrouve les grands points qui vont structurer les chapitres, et sur lesquels tu vas revenir en conclusion/perspectives).




\section{Estimation du risque inondation}
		\subsection{Concept de période de retour}
		Cours Benjamin + Thèse brigode 
		
	    \subsection{Estimations empiriques}
	    Cours Ben + biblio anciennes stedinger etc
	    \subsection{Ajustements statistiques}
	    cours ben + livres FFA : Hamed 2019; Jain 2019
	    ajuster une distribution : block maxima ou Over treshold
	    \subsection{Méthodes régionales}
	    Thèse pierre
	    \subsection{Méthodes pluie/débit}
	    thèse pierre
	    \paragraph{} SHYREG, GRADEX, SHADEX, etc
	    
	    \paragraph{} Quelle que soit la méthode d'estimation du risque retenue, des chroniques de débit sont nécessaires.
	    
\section{Élaboration des chroniques de débit}
		\subsection{Schéma hydrométrique usuel}
		charte de l'hydrométrie et refs internationnales (article JOH)
		\subsection{Quantification des incertitudes}
		uH, u jau, u RC, détarages, SPD
		
		\subsection{Hydrométrie en contexte historique}
		incertitude sur les débits de pointe histo : pas d'échelle\\
		difficulté d'avoir un échantillon exhaustif\\
		reconstitutions par la modélisation\\
	
		
\section{Analyse fréquentielle des crues}
		\subsection{Méthodes d'échantillonnage}
			\subsubsection{Max annuels}
			\subsubsection{Sup-seuil}
			
			ATTENTION AUX DOUBLONS AVEC SECTION 2
		\subsection{Sources d'incertitude}
			\subsubsection{Chroniques de débit}
			\subsubsection{Échantillonnage}
			\subsubsection{Choix de distribution}
			\subsubsection{Respect de l'hypothèse de stationnarité}
			
Sharma et al, 2018 : Despite evidence of increasing precipitation extremes, corresponding evidence for increases in flooding remains elusive. 
		
\section{Analyse fréquentielle des crues en contexte historique}
		\subsection{Intérêt de l'utilisation des données historiques}
		
		Illustration avec l'exemple de l'Ahr en 2021 : Ludwig et al, 2023 -> Figure 7
		"Again, this underlines the challenges of extreme value
statistics and the large uncertainties when estimating return
periods for the 2021 event. It also indicates the need for
even longer historical time series and reconstructions as far
as possible and/or the examination of the completeness of
the events between 1804 and 1946 as well as before 1804,
where there is evidence that over 70 floods occurred in the
Ahr river basin since the year 1500, including the large 1601
event (Seel, 1983). In addition, 1818 and 1848 were also
large events with currently no reconstructed streamflows."
		
		\subsection{Données continues anciennes}
		Nombreuses données qui dorment dans les archives : illustrations ? Données dispo en banque hydro VS réalité
		\subsection{Données pré-enregistrements continus}
		Liste d'études histo dans le monde, en europe et en France
		\subsection{Incertitudes autour de l'analyse fréquentielle historique}
		
\section{Le risque inondation dans la basse vallée  du Rhône}

\section{Organisation du manuscrit}






\printbibliography
\end{document}
